\chapter{\label{intro}Introduction}
% \begin{itemize}
%     \item Introduce the concept of topological insulators by drawing examples from Ando's review
%     \item State the first prediction of magnon hall effect
%     \item Onose experimental observation
%     \item Topological nature (murakami, mook, seshadri)
% \end{itemize}
Band theory has been successful in explaining many properties of solids, like room temperature conductivity of metals, by using the \textit{independent electron approximation.} This approximation works well due to the fact that at room temperature, kinetic energy terms in the Hamiltonian describing the system, dominates over the electron-electron Coulomb interaction term. This physical picture of independent particles, however, fails to explain collective phenomena at low temperatures. One such phenomenon is the existence of \textit{magnons}, which are quantized low energy excitation around the ordered magnetic ground state.
 
To explain collective excitation of spins at low temperatures, it suffices to consider
a model in which the atoms are fixed at their equilibrium position and the Hamiltonian consists of spin-degrees of freedom only. The dynamics of spin on a particular lattice is affected by the spins at its nearest neighbour via \textit{exchange interaction}. This kind of spin-only interaction tends to order the classical spin vectors; resulting in a ordered ground state at zero temperature. Any fluctuation at a particular site is distributed over the lattice because of this interaction and these constitute the so-called \textit{spin waves}. The quantization of energies of spin-waves suggests us to treat the exciation quanta as a quasiparticle - \textit{particle} because they have a definite momentum (like electrons) and localization. 

Recently, there has been a surge of activities in trying to understand the dynamics of magnons due to the possibility of development of devices on spin degrees of freedom. Spin current, the fundamental object of interest in spintronics research, is carried by magnons in magnetic insulators. Being a neutral quasi-particle, it is not expected that magnons exhibit any Hall effect. However, in 2010 it was suggested\cite{PhysRevLett.104.066403} that thermal Magnon Hall effect is possible in prinicple and a few months later it was indeed experimentally verified\cite{Onose297} on a 3D Pyrochlore lattice. 

....topological nature of magnon hall effect....shindou, murakami

\section{Berry phase}
------ formal derivation ---------
\section{TKNN invariant}
------ Ando's review first chern index only ------

recent results.
 
\setcounter{equation}{0}
\setcounter{table}{0}
\setcounter{figure}{0}
%\baselineskip 24pt


    



